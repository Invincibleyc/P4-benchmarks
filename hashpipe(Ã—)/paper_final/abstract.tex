\begin{abstract}

% doing this entirely in the dataplane
Identifying the ``heavy hitter'' flows or flows with large traffic volumes in
the data plane is important for several applications \emph{e.g.,} flow-size
aware routing, DoS detection, and traffic engineering.
%Understanding which  traffic flows consume the most bandwidth is useful for load
%balancing, detecting DoS attacks, and traffic engineering. Identifying these
%``heavy hitter'' flows in the data plane enables applications that distinguish
%between packets belonging to heavy and light flows at packet-processing time,
%\emph{e.g.,}
%to route heavy flows or DoS traffic differently. 
However, measurement in the data plane is constrained by the need for line-rate processing (at 10-100Gb/s) and
limited memory in switching hardware. We propose HashPipe, a heavy hitter detection algorithm using emerging programmable data planes.
HashPipe implements a pipeline of hash tables which retain counters for heavy
flows in the tables while evicting lighter flows over time.
We prototype HashPipe in P4 and
evaluate it with packet traces from an ISP backbone link and a data center. On
the ISP trace, we
find that
HashPipe identifies 95\% of the 300 heaviest flows with less than 80KB of memory
on a trace that contains 400,000 flows.
\end{abstract}
