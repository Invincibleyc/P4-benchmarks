\subsection{Existing Solutions}\label{sec:related}

The problem of finding heavy flows in a network is an instance of the ``frequent
items'' problem, which is extremely well studied in the streaming algorithms
literature~\cite{hh-cormode-survey}.
%
While high accuracy and low overhead are essential to any heavy hitter detection
approach, we are also specifically interested in implementing these algorithms
within the constraints of emerging programmable switches
(\Sec{sec:intro}).
%
We classify the approaches into two broad categories, {\em sampling} and {\em
  streaming,} discussed below.
%% \jrex{I think it'd be better to start with sketches like count-min sketch that
%%   make no effort to thin out or evict small flows, so you can transition from
%%   that to schemes that do such as count-min-with-cache, sample-and-hold, and
%%   finally space-savings.}

\NewPara{Packet sampling} using NetFlow~\cite{cisco-netflow} and sFlow~\cite{sflow}
is commonly implemented in routers today. These technologies record a subset of
network packets by sampling, and send the sampled records to collectors for
analysis. To keep packet processing overheads and data collection bandwidth low,
NetFlow configurations in practice use aggressively low sampling probabilities,
\eg 1\% or even 0.01\% \cite{sflow-sampling-rate, netflow-sampling-rate}. Such
undersampling can impact the estimation accuracy.

Sample and hold \cite{estan2002new} enhances the accuracy of packet sampling by
keeping counters for {\em all} packets of a flow in a ``flow table,'' once a
packet from that flow is sampled. However, implementing the flow table is
challenging, since the flow table requires a memory that is %sufficiently fast (to be read/written to by each packet) 
sufficiently large (to hold the necessary flow
entries).

Designing a large flow table\footnote{Large exact-match lookups are typically
  built with SRAM, as large TCAMs are expensive.} for fast packet {\em lookup}
is well-understood: hash table implementations are already common in switches,
for example in router forwarding tables and NetFlow~\cite{li2016flowradar}.
%
However, it is challenging to handle hash collisions when {\em adding} flows to
the flow table, when a packet from a new flow is sampled.
%
Some custom hardware solutions combine the hash table with a ``stash'' that
contains the overflow, \ie colliding entries, from the hash
table~\cite{sram-tcam-learning}. But this introduces complexity in the switch
pipeline, and typically involves the control plane to add entries into
the stash memory.
%
Ignoring such complexities, we evaluate sample and hold in \Sec{sec:evaluation}
by liberally allowing the flow table to lookup packets {\em anywhere} in a {\em
  list} that can grow up to a fixed size.

\NewPara{Streaming algorithms} implement data structures with bounded memory size
and processing time per packet, while processing {\em every packet} in a large
stream of packets in one pass. The algorithms are designed with provable
accuracy-memory tradeoffs for specific statistics of interest over the
packets. While these features make the algorithms attractive for network
monitoring, the specific algorithmic operations on each packet determine
feasibility on switch hardware.

Sketching algorithms like count-min sketch~\cite{cormode2005improved} and count
sketch~\cite{charikar2002finding} use per-packet operations such as hashing on
packet headers, incrementing counters at hashed locations, and determining the
minimum or median among a small number of the counters that were hashed
to. These operations can all be efficiently implemented on switch
hardware~\cite{yu2013software}.
%
However, these algorithms do not track the flow identifiers of packets, and hash
collisions make it challenging to ``invert'' the sketch into the constituent
flows and counters.
%
Simple workarounds like collecting packets that have high flow count estimates
in the sketch could result in significant bandwidth overheads, since most
packets from heavy flows will have high estimates.
%

Techniques like group testing~\cite{group-testing}, reversible sketches
\cite{schweller2004reversible}, and FlowRadar~\cite{li2016flowradar} can decode
keys from hash-based sketches. 
%
However, it is challenging to read off an accurate counter value for a packet in
the switch pipeline itself since the decoding happens off the fast packet-processing
path.
%
%
%Further, all the above approaches use the available memory to estimate {\em all}
%flows---not just the heavy ones.
%%(i) take up too many counter updates per packet
%%(ii) use too many counters overall too! accuracy-memory tradeoff not great

Counter-based algorithms~\cite{misra1982finding,manku2002approximate} focus on measuring the heavy
items, maintaining a table of flows and corresponding counts. They employ
per-counter increment and subtraction operations, but potentially all counters
in the table are updated during some flow insertions.
Updating multiple counters in a single stage is challenging within the deterministic time budget for each packet.
%Such worst-case table updates are not possible within the deterministic time budget for each packet.

\Spacesaving~\cite{metwally2005efficient} is a counter-based algorithm
that only uses $O(k)$ counters to track $k$ heavy flows, achieving the best
memory usage possible for a deterministic heavy-hitter algorithm. \Spacesaving
only updates one counter per packet, but requires finding the item with the
minimum counter value in the table. Unfortunately, scanning the entire table on
each packet, or finding the minimum in a table efficiently, is not directly
supported on emerging programmable hardware (\Sec{sec:intro}). Further,
maintaining data structures like sorted linked
lists~\cite{metwally2005efficient} or priority queues~\cite{pifo} requires
multiple memory accesses within the per-packet time budget. %% Recently proposed
%% advances in switch hardware may make such data structures
%% possible~\cite{pifo}.
%
However, as we show in \Sec{sec:algorithm}, we are able to adapt the key ideas
of the \spacesaving algorithm and combine it with the functionality of emerging
switch hardware to develop an effective heavy hitter algorithm.
%adapts the key ideas of
%the space saving algorithm and leverages emerging switch hardware with simple
%data structures.
