\section{Related Work}
\label{sec:other-related}

\NewPara{Applications that use heavy hitters.} Several applications
use information about heavy flows to do better traffic
management or monitoring. DevoFlow~\cite{devoflow} and
Planck~\cite{planck} propose exposing heavy flows with low overhead as ways to
provide better visibility and reduce congestion in the
network. UnivMon~\cite{univmon} uses a top-$k$ detection sketch internally as a
subroutine in its ``universal'' sketch, to determine more general statistics
about the network traffic. There is even a P4 tutorial application on switch
programming, that performs heavy hitter detection using a count-min-sketch-like
algorithm~\cite{p4-sigcomm-tutorial-hh-example}.

\NewPara{Measuring per-flow counters.} Several prior works like
FlowRadar~\cite{li2016flowradar}, CounterBraids~\cite{counterbraids}, and
others, \eg~\cite{li-randomized-counter-sharing} have proposed schemes to
measure accurate per-flow traffic counters. Along similar lines, hashing schemes
like cuckoo hashing~\cite{cuckoo-hashing} and d-left
hashing~\cite{vocking2003asymmetry} can keep per-flow state memory-efficiently,
while providing fast lookups on the state. Our goal is not to measure or keep
all flows; just the heavy ones. We show (\Sec{sec:evaluation}) that \TheSystem
uses 2-3 orders of magnitude smaller memory relative to having per-flow state
for all active flows, while identifying more than 95\% of the heavy flows.

\NewPara{Other heavy-hitter detection approaches.} The multi-resolution tiling
technique in ProgME~\cite{progme}, and the hierarchical heavy-hitter algorithm
of Jose \etal~\cite{jose2011online} solve a similar problem as
ours. They estimate heavy hitters in traffic {\em online}, by iteratively
refreshing the set of packet counters on a switch and ``zooming in'' to the
portions of traffic which are more likely to contain heavy flows. However, these
algorithms require the involvement of the control plane to run their
flow-space-partitioning algorithms, while \TheSystem works completely within the
switch pipeline. Further, both prior approaches require temporal stability of
the heavy-hitter flows to detect them over multiple intervals; \TheSystem
determines heavy hitters using the counters updated within the same interval.

%%- https://www.cise.ufl.edu/~sgchen/paper/sigmetrics2015.pdf
%% frameworks/apps: opensketch
%% packet mirroring: everflow, netsight
%% other monitoring: latency/loss measurement. LDA, reference latency
%%   interpolation, etc.
%% other apps: flow size distribution? billing?
%% other hash-based schemes? see survey of kirsch, varghese and mitzenmacher

%% "Our approach efficiently allocates the available space to the heavy hitters by either evicting a large number of the small flows on a packet-by-packet basis or restricting them to a small subset of the entire table space. This proves to be a crucial advantage in estimating the heavy hitters more accurately when compared to other existing schemes"


%% Group Testing - larger number of counters - many more tables as k increased - too much space to less important items - key recoverable
%% Space-Saving is the most space-efficient way of maintaining information

%% Really bad space-complexity they devote too much space to less imp items
%% Or major background processing to be able to scan or clean or evict items once in a while - amortized O(1)
%% We want something that is deterministically O(1)

%% Candidates:
%% - count-min: no keys here; only do top-k (no threshold)
%% - count-min + reversible sketch -- [XXX: why is this inaccurate?]. decoding overhead
%% - sample and hold: [XXX: inaccurate for the same amount of memory... but why?]
%% --> can we use the bounds from these papers?
%% - univmon?
%% - flowradar?
%% - other streaming heavy hitters algorithms (e.g., space savings) --> can't do on general purpose switch hardware
%% - discuss the fact that space saving achieves the lower bound of O(k)
